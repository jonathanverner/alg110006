\chapter*{Úvod}
V situaci, kdy je k dispozici množství velmi dobrých úvodních i pokročilých
učebnic programování se může zdát zbytečné psát učebnici novou. Chtěl bych
proto hned na úvod podotknout, že při psaní těchto skript jsem neměl ambici
napsat ``nejlepší'' učebnici; dokonce jsem ani neměl ambici napsat učebnici
``originální''. Skripta vznikala v podstatě jako příprava na letní semestr
přednášky ``Úvod do programování'' pro první ročník studentů logiky na 
Filozofické fakultě UK. Jejich hlavním přínosem je, že pokrývají --- víceméně 
přesně --- přednesenou látku. Přestože jsou skripta cílena zejména na 
mé studenty, doufám, že i případnému jinému čtenáři budou k prospěchu,
vyprovokují v něm zájem o informatiku a třeba ho povedou k další četbě jistě
mnohem lepších textů. Z nepřeberného výběru bych zde chtěl vyzdvihnout dvě
knížky --- skvělou českou knížku P.~T\"opfera: \emph{Algoritmy a programovací techniky}
(\cite{Topfer:1995}) a klasiku oboru \emph{The Art of Computer Programming} od
D.~Knutha.

Na závěr úvodu patří poděkování. Chtěl bych předně poděkovat své ženě Anše
za cenné připomínky (a její lásku, ale to je jiný příběh$\ldots$), Tomáši
Lavičkovi za upozornění na chyby v přiložených algoritmech a konečně také
Evropské unii, která v rámci projektu OPVK CZ.1.07/2.2.00/28.0216 
\emph{Logika: systémový rámec rozvoje oboru v ČR a koncepce logických 
propedeutik pro mezioborová studia} podpořila psaní těchto skript. O tento
projekt se na naší katedře skvěle starali Marta Bílková a Michal Dančák. 
A nakonec bych chtěl poděkovat tomu, který nás stvořil a vybavil schopností 
tvořit světy virtuální.

\hfill \emph{Autor}



\ifx\ucebnice\undefined
\renewcommand{\refname}{\textbf{Literatura}}
\bibliographystyle{mujstyl}
\bibliography{ref}

\end{document}
\fi